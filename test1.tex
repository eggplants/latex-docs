\documentclass[a4j,12pt]{jreport}
\title{ {\LaTeX} 動作確認テスト・サンプルファイル}
\author{情報リテラシTA}
\date{\today}
\begin{document}
\maketitle


\chapter{\LaTeX の世界にようこそ!}

\section{インストール成功!}
\LaTeX の世界にようこそ!この文章が「dviout」というソフトで閲覧できていれば、
インストールに成功しています。

\LaTeX(ラテフ)もしくは\TeX(テフ)は、
組版処理を行うソフトウェアです。
数学者・コンピュータ科学者のドナルド・クヌース氏によって作られました。

このソフトを使うと、きれいな文章の作成ができます。実際に出版の現場でも使われているそうです。
数学者が作ったということもあって、特に数式の出力がきれいにできるのが特徴です。
\begin{eqnarray}
& \displaystyle \lim _{x \rightarrow 1} \left( \frac{2}{x-1} - \frac{x+5}{x^3 -1} \right)\; ,\; 
& \displaystyle \int ^\pi _0 \cos ^2 (x)dx \nonumber
\end{eqnarray}
2つの数式が、きちんと表示されていますか?
複雑な数式が入った文章も、きれいに出力することができます。



\section{基本手順}

では、\LaTeX で文章を作る際の、基本的な手順をここに示します。


\begin{enumerate}
 \item ソースファイルをTeraPadなどのエディタで作成する。
 
 ソース(素)となるファイルを作成します。これにはエディタと呼ばれるソフトを使います。
 この地点では文章の形にはなっていません。
 
 \item ソースファイルをコンパイルして、dviファイルを作成する。
 
 パソコンに変換を命令して、先ほどつくったソースファイル
 をdviファイルに変換、文章の形にして確認します。
 
 \item dviファイルができたことを確認したら、PDFに変換する。
 
 dvi形式は、あまり一般的ではありません。
 そこで、Adobe Readerなどで閲覧ができるPDF形式に変換します。
 
 
\end{enumerate}


\end{document}
